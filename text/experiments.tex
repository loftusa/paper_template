\section{Experiments}
\label{experiments}

A set of experiments that usefully explore how good/interesting the method/model/central investigation of the paper is, and which tell a clear and cohesive story.

Results tables with lots of numbers and some of them \textbf{bold}.

Options for things to measure:
\begin{itemize}
    \item runtime
    \item sensitivity to important parameters
    \item scalability in various aspects: data size, problem complexity, etc
\end{itemize}

\begin{table}[ht]
\label{table:reconstruction_table}
\centering
\begin{tabular}{lccc}
\hline
Metric1 & Metric2 & Metric3 & Metric4 \\
\hline
Method1 (ours) & \textbf{0.011834} & \textbf{57.287} & 0.99965 \\
Method2 & 0.013030 & 56.450 & \textbf{0.99959} \\
Method3 & 0.034077 & 47.778 & 0.99569 \\
\end{tabular}
\caption{Results table with lots of numbers and some of them bold}
\end{table}

Options for things to show:
\begin{itemize}
    \item Absolute performance
    \item Relative performance to naive approaches
    \item Relative performance to previous approaches
    \item Relative performance among different proposed approaches
\end{itemize}

\begin{figure}[h]
\centering
\begin{minipage}{0.3\linewidth}
    \centering
    % If you want to frame the image, uncomment the next line and comment out the original \includegraphics line
    % \fbox{\includegraphics[width=\linewidth]{figs/figure.png}}
    \includegraphics[width=\linewidth]{figs/systemfigure.png} 
\end{minipage}
\caption{System figure clearly showing technical details of measurements made by experiments.}
\label{fig:system}
\end{figure}