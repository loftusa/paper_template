\documentclass{article}


% if you need to pass options to natbib, use, e.g.:
%     \PassOptionsToPackage{numbers, compress}{natbib}
% before loading neurips_2023


% ready for submission
% \usepackage{neurips_2023}


% to compile a preprint version, e.g., for submission to arXiv, add add the
% [preprint] option:
    \usepackage[preprint]{neurips_2023}


% to compile a camera-ready version, add the [final] option, e.g.:
%     \usepackage[final]{neurips_2023}


% to avoid loading the natbib package, add option nonatbib:
%    \usepackage[nonatbib]{neurips_2023}


\usepackage[utf8]{inputenc} % allow utf-8 input
\usepackage[T1]{fontenc}    % use 8-bit T1 fonts
\usepackage{hyperref}       % hyperlinks
\usepackage{url}            % simple URL typesetting
\usepackage{booktabs}       % professional-quality tables
\usepackage{amsfonts}       % blackboard math symbols
\usepackage{nicefrac}       % compact symbols for 1/2, etc.
\usepackage{microtype}      % microtypography
\usepackage{xcolor}         % colors
\usepackage{enumitem}
\usepackage{graphicx}

%\donotusepackage{fullpage} --> Steinhardt says this package causes .sty problems for many ML conferences.

% Taking various advice about paper-writing from people who are good at writing papers and condensing it into a template. Inspiration from: 
  % Jennifer Widom, Stanford - https://cs.stanford.edu/people/widom/paper-writing.html
  % Karpathy - https://karpathy.github.io/2016/09/07/phd/
  % Jacob Steinhardt - https://bounded-regret.ghost.io/advice-for-authors/

% Be ambitious with the problem. a 10x more impactful and important problem is often at most only 2-3 times harder to solve. 10x thinking forces you to think out of the box, to confront the real limitations of an approach, to think from first principles, to change the strategy completely, to innovate. Aspire to improve something by 100\% or more. - Karpathy

%%% BEGIN DOCUMENT
\title{Middle-of-the-road length title, plus a fun name that sticks in people's minds}
\author{%
Alexander Loftus\thanks{Equal contribution. Correspondence to \texttt{alexl@northeastern.edu}.} \\
Northeastern University \\
\And
David Bau$^*$ \\
Northeastern University \\
}

\begin{document}
\maketitle
\addtolength{\tabcolsep}{-5pt}

% beginning of paper
\begin{abstract}
\label{abstract}

% https://bounded-regret.ghost.io/advice-for-authors/

The first sentence/phrase should be something that all readers agree with. The second should be something that many readers would find surprising, or wouldn't have thought about before; but it should follow from (or be supported by) the first sentence. Warm the reader up with context. State the problem, your approach and solution, and the main contributions of the paper. Include little if any background and motivation. Be factual but comprehensive. The material in the abstract should not be repeated later word for word in the paper.

This paper can be found at \url{https://accompanyingwebsite.baulab.info/}.

\end{abstract}
\section{Introduction}
\label{introduction}

% coherent top level narrative
% describing what I want to do in comments
% before I write the paper
% to organize thoughts

% - avoid using vague phrases like "much recent interest" and "increasingly important"
% - provide context before introducing a new concept. 
% - don't beat around the bush; if the point is A -> B, then say that, rather than just being humble and pointing out A.

Roughly one page total. One paragraph for each of the below. Organize each paragraph around a single concrete point stated in the first sentence that is then supported in the rest of the paragraph.

\begin{enumerate}
\item What is the problem?
\item Why is it interesting and important?
\item Why is it hard? (E.g., why do naive approaches fail?)
\item Why hasn't it been solved before? (Or, what's wrong with previous proposed solutions? How does mine differ?)
\item What are the key components of my approach and results? Also include any specific limitations.
\end{enumerate}

\subsection{Summary of Contributions}
\label{contributionsummary}
% - tell the reader why they should care about a paper right away, possibly even doing it explicitly

The major contributions of this paper are:

\begin{itemize}
    \item Contribution 1
    \item Contribution 2
    \item Contribution 3
\end{itemize}

The rest of the paper is structured as follows: Section \ref{sec:contribution1} describes Contribution 1. Section \ref{sec:contribution2} explains Contribution 2. Finally, Contribution 3 is presented in Section \ref{sec:contribution3}.

\begin{figure}[h]
\centering
\begin{minipage}{0.55\linewidth}
    \centering
    % If you want to frame the image, uncomment the next line and comment out the original \includegraphics line
    % \fbox{\includegraphics[width=\linewidth]{figs/figure.png}}
    \includegraphics[width=\linewidth]{figs/pullfigure.png} 
\end{minipage}
\caption{Well-designed pull/hook figure on page 1 or 2 to bring the reader in and convey the central point of the paper}
\label{fig:system}
\end{figure}
\section{Related Work}
\label{related}

Roughly 1 page. Good density of citations \cite{alex1994example} - not too sparse but not too crowded.

Keep this here if related work can be short yet detailed enough, or if it's critical to take a strong defensive stance about previous work right away. In this case Related Work can be either a subsection at the end of the Introduction, or its own Section 2.

Put this section at the end if it can be summarized quickly early on (in the Introduction or Preliminaries), or if sufficient comparisons require the technical content of the paper. In this case Related Work should appear just before the Conclusions, possibly in a more general section "Discussion and Related Work".

% Main section describing model/method and core contribution
\section{First section}
\label{body}

Organize the paper around a single core contribution with surgical precision. Do not add additional fluff.

Some general guidelines:

\textbf{Guideline \#1}: A clear new important technical contribution should have been articulated by the time the reader finishes page 3 (i.e., a quarter of the way through the paper).

\textbf{Guideline \#2}: Every section of the paper should tell a story. (Don't, however, fall into the common trap of telling the entire story of how you arrived at your results. Just tell the story of the results themselves.) The story should be linear, keeping the reader engaged at every step and looking forward to the next step. There should be no significant interruptions -- those can go in the Appendix; see below.

Aside from these guidelines, which apply to every paper, the structure of the body varies a lot depending on content. Important components are:

\begin{itemize}

    \item \textbf{Running Example}: When possible, use a running example throughout the paper. It can be introduced either as a subsection at the end of the Introduction, or its own Section 2 or 3 (depending on Related Work).
    \item \textbf{Preliminaries}: This section, which follows the Introduction and possibly Related Work and/or Running Example, sets up notation and terminology that is not part of the technical contribution. One important function of this section is to delineate material that's not original but is needed for the paper. Be concise -- remember Guideline \#1.
    \item \textbf{Content}: The meat of the paper includes algorithms, system descriptions, new language constructs, analyses, etc. Whenever possible use a "top-down" description: readers should be able to see where the material is going, and they should be able to skip ahead and still get the idea.
\end{itemize}

\subsection{method explanation}
Method/model description with technical section somewhere with math details.

\subsection{more sections of the paper}
Explain what is being done in the section. Explain what the core challenges are. Explain what a baseline approach is or what others have done before. Motivate and explain what I'm doing

\subsection{more sections of the paper}
Explain what is being done in the section. Explain what the core challenges are. Explain what a baseline approach is or what others have done before. Motivate and explain what I'm doing
\section{Experiments}
\label{experiments}

A set of experiments that usefully explore how good/interesting the method/model/central investigation of the paper is, and which tell a clear and cohesive story.

Results tables with lots of numbers and some of them \textbf{bold}.

Options for things to measure:
\begin{itemize}
    \item runtime
    \item sensitivity to important parameters
    \item scalability in various aspects: data size, problem complexity, etc
\end{itemize}

\begin{table}[ht]
\label{table:reconstruction_table}
\centering
\begin{tabular}{lccc}
\hline
Metric1 & Metric2 & Metric3 & Metric4 \\
\hline
Method1 (ours) & \textbf{0.011834} & \textbf{57.287} & 0.99965 \\
Method2 & 0.013030 & 56.450 & \textbf{0.99959} \\
Method3 & 0.034077 & 47.778 & 0.99569 \\
\end{tabular}
\caption{Results table with lots of numbers and some of them bold}
\end{table}

Options for things to show:
\begin{itemize}
    \item Absolute performance
    \item Relative performance to naive approaches
    \item Relative performance to previous approaches
    \item Relative performance among different proposed approaches
\end{itemize}

\begin{figure}[h]
\centering
\begin{minipage}{0.3\linewidth}
    \centering
    % If you want to frame the image, uncomment the next line and comment out the original \includegraphics line
    % \fbox{\includegraphics[width=\linewidth]{figs/figure.png}}
    \includegraphics[width=\linewidth]{figs/systemfigure.png} 
\end{minipage}
\caption{System figure clearly showing technical details of measurements made by experiments.}
\label{fig:system}
\end{figure}
\section{Conclusion}
\label{conclusion}

Put in the conclusion everything that wants to have gone into the introduction, but wouldn't have made sense for lack of context. Should be a short summarizing paragraph. Under no circumstances should the paragraph simply repeat material from the Abstract or Introduction. In some cases it's possible to now make the original claims more concrete, e.g., by referring to quantitative performance results. Discuss open questions you'd like other researchers to think about. Likely only the 5ish people most interested in your paper will actually read this section, so it's worth somewhat tailoring to that audience. Unfortunately, paper reviewers might also read this section, so you can't tailor it too much. This section is most important when submitting to NIPS or ICML, sometimes you can skip it in other conferences.
\section{Future Work}

Show how the work new research directions. Use bullet lists. A couple of things to keep in mind:

\begin{itemize}
    \item Say if we're actively engaged in follow-up work, say so. E.g.: "We are currently extending the algorithm to... blah blah, and preliminary results are encouraging." This statement serves to mark your territory.

    \item Conversely, be aware that some researchers look to Future Work sections for research topics. 
\end{itemize}
\section{Acknowledgements}

Acknowledge anyone who contributed in any way: through discussions, feedback on drafts, implementation, etc. If in doubt about whether to include someone, include them.

% Make all citations complete and consistent. Do not just copy random inconsistent BibTex (or other) entries from the web and call it a day. Check over the final bibliography carefully and make sure every entry looks right.
\bibliography{anthology}
\bibliographystyle{icml2022}

\section{Appendices}
Appendices should contain detailed proofs and algorithms only. Appendices can be crucial for overlength papers, but are still useful otherwise. Think of appendices as random-access substantiation of underlying gory details. As a rule of thumb:

\begin{itemize}
    \item Appendices should not contain any material necessary for understanding the contributions of the paper.
    \item Appendices should contain all material that most readers would not be interested in.
\end{itemize}

% general notes
% - always release code. Will dramatically increase citations and make code better due to fear of public shaming from bad code

% Grammar \ sentence notes
    % - define all terminology and notation in the opening section
    % - never say "for various reasons"
    % - void nonreferential use of "this" e.g. 'Our experiments test several different environments and the algorithm does well in some but not all of them. This is important because'
    % - italics are for definitions/quotes, not emphasis
    % - use 'which' versus 'that' correctly: 'that' is defining; 'which' is nondefining

% mechanics notes

    % - Always run a spelling checker on your final paper, no excuses.
    % - For drafts and technical reports use 11 point font, generous spacing, 1" margins, and single-column format. There's no need to torture your casual readers with the tiny fonts and tight spacing used in conference proceedings these days.
    % - In drafts and final camera-ready, fonts in figures should be approximately the same font size as used for the text in the body of the paper.
    % - Tables, figures, graphs, and algorithms should always be placed on the top of a page or column, not in the body of the text unless it is very small and fits into the flow of the paper.
    % - Every table, figure, graph, or algorithm should appear on the same page as its first reference, or on the following page (LaTex willing...).
    % - Before final submission or publication of your paper, print it once and take a look -- you might be quite surprised how different it looks on paper from how it looked on your screen (if you even bothered to look at it after you ran Latex the last time...).
    

\end{document}